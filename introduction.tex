\newenvironment{listage}{%
    \par%
    \medskip
    \leftskip=4em\rightskip=2em%
    \noindent\ignorespaces}{%
    \par\medskip}

\section{Contexte}
En l'état actuel, lorsqu'un les pompiers doivent intervenir dans le cas d'un incident impliquant des fuites d'hydrocarbures sur route, la répartition
du produit absorbant se fait via une remorque-semoirs dont les sections de dépose du produit sont télécommandées directement par un opérateur.
Cela signifie que durant l'intervention, une personne doit marcher à côté de la remorque afin d'observer et contrôler la répartition du produit, nous pouvons relever les problèmes suivant:
\begin{enumerate}
\item L'opérateur doit marcher: Limitation de la vitesse du véhicule, longues distances épuisantes.
\item La détection se fait à l'oeil nu: L'opérateur doit rester concentrer en permanance au risque de devoir repasser sur certaines zone.
\item La commande est manuelle: Possibilité d'erreur humaine sur la commande risquant la dépose de surplus de produit ou de devoir repasser sur certaines zone.
\end{listage}
%%if
\section{Citations et bibliographie}
Citer vos sources est essentiel. Avec \texttt{biblatex} vous pouvez facilement citer des articles, des livres ou des sites internet. Toutes les citations dans le texte seront automatiquement regroupées en fin de document dans la section \guillemotleft Bibliographie\guillemotright. Par exemple, citons un article d'Einstein \cite{einstein} ou le livre de Dirac \cite{dirac}.

Parfois il peut être utile d'utiliser un gestionnaire de bibliographie. La communauté académique recommande l'outil \href{https://www.zotero.org/}{Zotero} qui permet de gérer une bibliothèque numérique d'ouvrages et de références numériques. Il permet également de générer une bibliographie compatible avec \LaTeX.

Notez qu'il est très facile d'obtenir l'extrait \texttt{bibtex} depuis des journaux. Sélectionnez \emph{export/citation}. Si vous le pouvez choisissez \texttt{bibtex}. Dans le cas d'un format \texttt{.ris}, utilisez un convertisseur en ligne comme \href{http://www.bruot.org/ris2bib/}{ris2bib}.

\section{Adapter votre modèle}
Ce document n'est qu'un modèle ayant pour but de revoir les quelques avantages de \LaTeX~ et les fonctionnalités qui pourraient vous être utiles pour rédiger un rapport académique. N'hésitez pas à supprimer les parties inutiles et à adapter ce modèle à vos besoins.
%%fi