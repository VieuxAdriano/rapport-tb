\section{Communication}
Dans le cadre de ce projet, plusieurs éléments devront communiquer entre eux, notamment l'unité de traitement d'image,
l'actionnement de la télécommande ainsi que le retour vidéo pour le copilote, il est donc important de définir comment ils communiqueront entre eux.
Il est important de se rappeler que leurs emplacements seront les suivants:\\
\begin{table}[h]
    \begin{center}
        \caption{Liste des élements et emplacements}
        \begin{tabular}{|c|l|}
            Elements                        & Emplacement                                 \\ \hline
            Caméra et boitier de traitement & Barre de la remorque                        \\
            Actionnement de la télécommande & A proximité du rangement de la télécommande \\
            Affichage pour copilote         & Place passager du véhicule
        \end{tabular}
    \end{center}
\end{table}\\
Dans notre cas de figure, les élements sont séparés de plusieurs mètres, voir même dans un autre véhicule. La communication cablée n'est pas une solution.
\begin{itemize}
    \item Via une connexion Bluetooth.
    \item Via un réseau \Gls{wifi} local.
\end{itemize}
La connexion sans fil doit garantir une connectivité stable sur un rayon de 3 mètres autour du véhicule (avec obstacles). De plus, le débit doit être
suffisant pour diffuser un flux vidéo (entre 30 et 60 \Gls{fps}). La solution du réseau local \Gls{wifi} semble être la meilleure.\\
\textbf{La possibilité de communiquer via un réseau local \Gls{wifi} devient donc un prérequis pour la suite de la sélection du matériel.}
\section{Capture et traitement d'image}
\subsection{Eclairage}
La sélection de l'éclairage va dépendre des points suivants:
\begin{listage}
    1. La longueur d'onde nécessaire.\\
    2. L'intensité lumineuse nécessaire.\\
    3. La taille du champ à "éclairer".
\end{listage}
Le premier point s'est défini durant la phase de recherche, l'idéal serait d'avoir une source émettant une onde de 3300 \si{\nano\metre}.

\subsection{Caméra}
Détail..
\subsection{Système embarqué de traitement}
Détail..

\section{Système de commande}
\subsection{Actionneurs de la télécommande}
Les idées de base d'actionneurs permettant d'appuyer sur la télécommande sont les suivants:
\begin{listage}
    1. Vérins électriques.\\
    2. Servomoteurs.\\
    3. Moteurs pas à pas.
\end{listage}

Les vérins électriques ont l'avantages d'être facile à installer, il suffirait d'une plaque de support perforée au dessus de chaque bouton et d'y insérer
les pièces à la profondeur souhaitée, puis de les actionner/rétracter pour appuyer ou non sur un bouton. Cependant, les vérins disponibles sur le marché semblent être surdimensionnés pour la tâche en question, les forces et
consommation pour un fonctionnement normal sont très grand. Les prix sont également très élevé, il est difficile de trouver des pièces de moins de 100 CHF l'unité
pour les dimensions désirées. De plus, les délais de livraison durant la période dédiée au développement de se projet sont relativements incertains pour se genre de pièce.\\

L'idée, avec les servomoteurs, serait de les équiper d'un élement de préhension tangent à l'axe de rotation du moteur permettant de presser sur les boutons.
Le pilotage d'un servomoteur se fait par PWM, nécessitant donc qu'une seule sortie par élements au niveau de la carte de commande. De plus, il est assez facile de
trouver des servomoteurs avec des caractéristiques en phase avec le projet, c'est à dire d'un point de vu taille, force et consommation, et ce à des prix
raisonnable (avoisinant les 5 à 10 CHF). L'installation sur un support est plus complexe que pour les vérins.\\

L'utilisation du moteur pas à pas serait similaires au servomoteur, c'est à dire avec un élement qui presse sur les boutons.
Il est assez facile de trouver des moteurs avec des caractéristiques en phase avec le projet à des prix résonnable (avoisinnant 10 CHF). La commande d'un moteur pas à pas
se fait usuellement via un pilote servant d'interface entre le moteur et la carte de commande. Il est nécessaire d'avoir une carte pilote par moteur (environ 10 CHF par unité).
L'installation sur un support est plus complexe que pour les vérins.\\

Avec ces informations, en comparant prix, facilité d'acquisition, pilotage, caractéristique obtenable, délai de livraison, la solution du servomoteur semble être la plus
appropriée.\textbf{Je vais donc baser la suite du développement avec des servomoteurs comme actionneurs de la télécommande.}\\

\subsection{Pilotage des actionneurs}
La carte de commande servant à piloter les servomoteurs devra être capable de:
\begin{listage}
    1. Commander 6 servomoteurs simultanément.\\
    2. Communiquer (reçevoir des informations) avec le boitier de traitement d'image.\\
    3. Afficher un statut via un voyant (une led par exemple).
\end{listage}

Plusieurs possibilités se présentent , parmis la famille des Arduinos:
\begin{listage}
    1. Arduino Nano 6.\\ 20CHF
    2. Arduino Micro 7.\\ 25CHF
    3. Arduino Uno 6.\\ 25-30CHF
    4. Arduino Mega 15.\\ 40CHF
    5. Arduino MKR WiFi 1010 10. 40CHF
\end{listage}
Les 4 premières cartes présentent cochent 2 parmis les 3 prérequis de sélection. Ce qui les différentie c'est majoritairement le coup, l'encombrement et la vitesse de travail.
Au vu de la simplicité des tâches, un Nano ou Micro serait suffisant. En imaginant que la communication avec l'unité de traitement d'image se fasse via un réseau \Gls{wifi} local,
il faudrait ajouter un module spécifique permettant la connexion, par exemple l'ESP8266.\\
En plus de pouvoir commander les servomoteurs et d'afficher un statut, le MKR WiFi 1010 possède un module \Gls{wifi} préintégré.
Les prix entre les cartes avec l'ajout d'un module et le MKR WiFi 1010 sont plus ou moins semblable. Ce dernier offre l'avantage d'être compacte et de limiter le câblage.
\textbf{Je vais donc baser la suite du développement avec l'Arduino MKR WiFi 1010.}
\section{Retour vidéo}
Le résultat de l'analyse du traitement d'image et la mise en évidence des traces d'hydrocarbure doivent être visible par le copilote
