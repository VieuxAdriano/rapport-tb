\section{Planning}
Le planning complété avec les heures effectives est disponible en annexe \ref{planning}.

Les cases en bleues représentent les périodes durant lesquelles je pensais effectuer les tâches au début du projet. Il y a beaucoup de différences entre la planification et la réalité.
\begin{itemize}
    \item Certaines étapes dépendent de tâches précises, un décalage du planning en entraîne un autre.
    \item Certains événements ne se sont pas fait aux moments prévus, comme la première visite du semoir ou les réceptions de commandes.
    \item Certaines tâches complémentaires se déroulent différemment qu'imaginé, par exemple il a fallu beaucoup de temps pour définir l'éclairage, mais très peu pour choisir un capteur adapté.
    \item J'ai été très optimiste lors de l'estimation de l'effort à fournir pour certaines tâches.
\end{itemize}

\section{Etat du projet}
Un aperçu de l'état du cahier des charges:
\begin{itemize}
    \item \ding{51} Préparer un planning détaillant les tâches.
    \item Définir un setup permettant de:
          \begin{itemize}
              \item \ding{51} Analyser les traces d'hydrocarbures selon leur positionnement sur la route.
              \item \ding{55} Commander la télécommande du semoir.
          \end{itemize}
    \item \ding{51} Sélectionner les éléments composant le setup.
    \item \ding{51} Commander les éléments.
    \item \ding{51} Assembler les éléments.
    \item \ding{51} Tester le setup et procéder aux ajustements nécessaires.
    \item \ding{51} Développer un logiciel d'analyse et de gestion de la télécommande.
    \item \ding{51} Tester le logiciel d'analyse et de gestion de la télécommande.
    \item \ding{55} Valider le logiciel et le comportement sur la route.
    \item \ding{55} Analyser et interpréter les résultats.
\end{itemize}

Comme mentionné à plusieurs reprises, la méthode de préhension n'a pas été définie. L'ordre des événements et surtout les tests infructueux avec la méthode sélectionnée de base
m'ont fait prendre du retard sur la sélection et l'installation.

L'aspect est tout de même fonctionnel d'un point de vue soft, pour rappel les sorties sont à l'état haut lorsque le préhenseur doit être actif et à l'état bas lorsqu'il est en repos.
Le RPi4 étant déjà précâblé, il suffit de définir le système de préhenseur pour rapidement le faire fonctionner.

Je n'ai malheureusement pas eu le temps de faire des essais en situation réelle, il reste donc de potentiels ajustement à faire au niveau des programmes.

\section{Conclusion technique}
L'aspect mécanique n'est pas mon domaine, j'ai eu des difficulté à me rendre compte de ce qui était réalisable ou non.
Les divers fixation, plaques d'adaptations et supports peuvent certainement être améliorés.

En ce qui concerne les captures, il a fallu pas mal d'adaptation entre les essais en laboratoire et en extérieur, les conditions d'éclairage sont très différentes,
bien plus que ce que j'imaginais. C'était pour moi un premier pas dans le monde de l'infrarouge, je ne savais pas vraiment à quoi m'attendre.

Le système fonctionne en partie. Il capture des images selon les conditions et paramètres définis, l'image est traitée, les zones à traiter sont détectées,
les ouvertures des vérins sont définies et le RPi4 sort les signaux de commande. Il reste les préhenseurs à définir et tester.
\section{Conclusion personnelle}
Le début du projet est allé très vite, là ou mes collègues avaient du temps pour poser leurs idées, de mon côté il a fallu concilier réflexion et prise de décision.
C'était très vite frustrant si j'étais bloqué sur un problème quelques heures. J'ai pu garder mon calme et avancer correctement, nos entrevues avec M. Bressy m'ont permis de rester sur les rails.

C'était malgré tout agréable de travailler à plein temps sur ce projet, ça m'a préparé pour la suite de ma vie professionnelle.
\section{Remerciement}
Je tenais à remercier les différentes personnes m'ayant aidés durant ce projet.

Je remercie M. Tristan Lieberherr, qui m'a partagé ses connaissances sur les Raspberry Pi et sur la configuration des réseaux WiFi locaux.

Je remercie M. Nicolas Tzaut, qui m'a très gentiment cédé un morceau de route provenant de sa démolition, grâce auquel j'ai pu effectuer
des tests de capture d'image en laboratoire très rapidement après le début du projet.

Et je remercie M. Pierre Bressy, qui m'a suivi tout du long du projet. Il m'a conseillé, guidé et proposé de nouvelles idées lorsque
j'étais bloqué.
\vfil
\hspace{8cm}\makeatletter\@author\makeatother\par
\hspace{8cm}\begin{minipage}{5cm}
    %%if
    % Place pour signature numérique
    \printsignature
    %%fi
\end{minipage}