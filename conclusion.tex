\section{Ressenti personnel}
La ou en temps normal il y une semaine entre deux jours de travail permettant de réfléchir, là y'a rien, très vite frustrant si bloqué plus de quelques heures..

Gardé mon calme et avancé comme je pouvais, fréquence d'entre vue plus élevée permettait de rester sur les rails.

Edition du rapport en latex chaud pour une première

Donner ressenti en fonction de l'état du proto
\section{Conclusion technique}
Difficulté projection aspect mécanique
Condition de test très différent de condition extérieur, ne pas savoir avec exactitude ce qu'on verra

Concilier avance rapide et choix de composants

Découverte de l'électronique embarquée et de mécanique, pas super propre, mais prototype plus ou moins fonctionnel.

Donner l'état du proto en fin de projet fonctionnel ou pas tout à fait

Parler du planning, parler des différences, résultat en annexe.
\section{Remercicement}
Je tenais à remercier les différentes personnes qui m'ont entourés, accompagnés et aidés durant ce projet.

Je remercie M. Tristan Lieberherr, qui m'a partagé ses connaissances sur les Rapsberry Pi et sur la configuration des réseaux WiFi locaux.

Je remercie M. Nicolas Tzaut, qui m'a très gentiment céder un morceau de route provenant de sa démolition, grâce auquel j'ai pu effectué
des tests de capture d'image en laboratoire très rapidement après le début du projet.

Et je remercie finalement M. Pierre Bressy, qui m'a suivi tout du long du projet. Il a su me... et me donner de nouvelles idées lorsque
j'étais bloqué.
\vfil
\hspace{8cm}\makeatletter\@author\makeatother\par
\hspace{8cm}\begin{minipage}{5cm}
    %%if
    % Place pour signature numérique
    \printsignature
    %%fi
\end{minipage}

%remercier Tzaut & Fils Sa pour les matériaux, Tristan pour les apports Rpi4