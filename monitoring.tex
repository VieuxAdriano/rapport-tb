\section{Communication}
Il a été décidé durant la phase de recherche que l'affichage se fera via un écran "volant" qui recevera les images via un réseau WiFi local
hébergé par le Rapsberry Pi. Les étapes pour y arriver sont les suivantes:
\begin{enumerate}
    \item Avoir flashé avec l'OS Rapsberry (en utilisant Rapsberry imager ou BalenaCloud).
    \item Être connecté à internet.
    \item Télécharger le package 'hostapd': \textbf{sudo apt install hostapd}.
    \item Télécharger le package 'dnsmasq': \textbf{sudo apt-get install dnsmasq}.
    \item Atteindre le fichier conf suivant: \textbf{sudo nano /etc/hostapd/hostapd.conf}.
    \item Y copier le code de la figure \ref{config_hostapd}, sauver + quitter: ctrl+X, puis Y, puis Enter.
    \item Atteindre le fichier suivant: \textbf{sudo nano /etc/default/hostapd}.
    \item Y copier le code de la figure \ref{default}, sauver + quitter: ctrl+X, puis Y, puis Enter.
    \item Atteindre le fichier suivant: \textbf{sudo nano /etc/dhcpcd.conf}.
    \item Y copier le code de la figure \ref{dhcpcd}, sauver + quitter: ctrl+X, puis Y, puis Enter.
    \item Redémarrer le Rpi4 avec: \textbf{sudo reboot}.
\end{enumerate}

\begin{listing}[h]
    \inputminted{makefile}{assets/figures/hostapd.make}
    \caption{Configuration hostapd \label{config_hostapd}}
\end{listing}

\begin{listing}[h]
    \inputminted{makefile}{assets/figures/default.make}
    \caption{Lancement du réseau au démarrage \label{default}}
\end{listing}

\begin{listing}[h]
    \inputminted{makefile}{assets/figures/dhcpcd.java}
    \caption{configuration dhcpcd \label{dhcpcd}}
\end{listing}

Avec l'ajout de ces lignes, le Rpi4 emet un réseau local dès son démarrage. Il est possible de s'y connecter avec n'importe quel
type d'appareil. Le nom du réseau et le mot de passe sont paramétrables, actuellement, c'est:
\begin{itemize}
    \item Nom du réseau: \textbf{SemoirSDIS}
    \item Mot de passe: \textbf{semoirSDIS}
\end{itemize}\\
À noter qu'il est obligatoire de s'y connecter pour avoir le retour vidéo.
\section{Affichage}
Il existe des codes sources disponibles en ligne permettant d'afficher le retour caméra de marque Rapsberry Pi.
Je me suis inspiré de celui-ci pour mon affichage: \cite{code_camera}.\\
\textbf{Mettre screenshot du rendu sur natel/pc}\\
\section{Accès}
L'accès au retour vidéo se fait via une page locale dont l'accès nécessite d'être connecté au WiFi émis par le Rpi4.
La page à atteindre est la suivante:
\begin{itemize}
    \item  xxx.xxx.xx.xx
\end{itemize}
Ou avec le QR code suivant: METTRE QR CODE
