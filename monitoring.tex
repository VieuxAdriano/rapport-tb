\section{Communication}
Il a été décidé durant la phase de recherche que l'affichage se fera via une page locale qui recevra les images via un réseau WiFi local
hébergé par le Raspberry Pi \cite{soren_how-_2013}. Les étapes pour y arriver sont les suivantes:
\begin{enumerate}
    \item Flasher avec l'OS Raspberry (en utilisant Raspberry imager ou BalenaCloud).
    \item Être connecté à internet.
    \item Télécharger le package 'dnsmasq': \textbf{sudo apt-get install dnsmasq}.
    \item Atteindre le fichier suivant: \textbf{sudo nano /etc/wpa\char`_supplicant/wpa\char`_supplicant.conf}.
    \item Y ajouter le code du listing \ref{wpa}, sauver + quitter: ctrl+O, puis ctrl+X.
    \item Atteindre le fichier suivant: \textbf{sudo nano /etc/network/interfaces}.
    \item Y remplacer le code du listing \ref{interfaces}, sauver + quitter: ctrl+O, puis ctrl+X.
    \item Atteindre le fichier suivant: \textbf{sudo nano /etc/dnsmasq.conf}.
    \item Y remplacer le code du listing \ref{dnsmasq}, sauver + quitter: ctrl+O, puis ctrl+X.
    \item Redémarrer le Rpi4 avec: \textbf{sudo reboot}.
    \item Couper la connexion réseau avec: \textbf{sudo ifdown wlan0}.
    \item Accéder à la liste des réseaux avec: \textbf{sudo wpa\char`_cli -i wlan0 list\char`_networks}, puis saisir le réseau avec: \textbf{sudo wpa\char`_cli -i wlan0 select\char`_network 1} (exemple figure \ref{networks_list})
    \item Sauver la nouvelle configuration avec \textbf{sudo wpa\char`_cli -i wlan0 save\char`_config}.
    \item Relancer la connexion réseau avec: \textbf{sudo ifup \underline{wlan0=ap0}}.
\end{enumerate}
\begin{listing}[ht]
    \inputminted{makefile}{assets/figures/wpa_supplicant.make}
    \caption{Configuration wpa\char`_supplicant \label{wpa}}
\end{listing}

\begin{listing}[ht]
    \inputminted{makefile}{interfaces.make}
    \caption{Configuration de l'interface réseau \label{interfaces}}
\end{listing}

\begin{listing}[ht]
    \inputminted{makefile}{dnsmasq.make}
    \caption{Configuration dnsmasq \label{dnsmasq}}
\end{listing}

\begin{figure}[H]
    \centering
    \includegraphics[width=13cm]{assets/figures/network_list.PNG}
    \caption{Liste de networks \label{networks_list}}
\end{figure}

Avec l'ajout de ces lignes, le Rpi4 émet un réseau local, pour que l'émission se fasse dès le démarrage de la carte,
il faut ouvrir le fichier suivant avec : \textbf{sudo nano /home/tb/.bashr} et y ajouter les lignes de codes des étapes \textbf{11 et 14}.

Il est possible de s'y connecter avec n'importe quel
type d'appareil. Le nom du réseau et le mot de passe sont paramétrables, actuellement, c'est:
\begin{itemize}
    \item Nom du réseau: \textbf{SemoirSDIS}
    \item Mot de passe: \textbf{123456789}
\end{itemize}\\
À noter qu'il est obligatoire de s'y connecter pour avoir le retour vidéo.

\begin{figure}[H]
    \centering
    \includegraphics[width=3cm]{assets/figures/acces_wifi.PNG}
    \caption{QR code - Connexion WiFi local}
\end{figure}

\newpage
\section{Affichage}
Il existe des codes sources disponibles en ligne permettant d'afficher le retour caméra basé sur le package Picamera2 \cite{picamera2}.
Je me suis inspiré de celui-ci pour mon affichage: \cite{code_camera}.

L'utilisation de \textbf{Picamera2} nécessite l'installation de dépendances:
\begin{itemize}
    \item \textbf{sudo apt-get install libatlas-base-dev}.
    \item \textbf{sudo apt install -y python3-picamera2}.
\end{itemize}

\begin{figure}[H]
    \centering
    \includegraphics[width=13cm]{assets/figures/retour_video_pc.png}
    \caption{Retour vidéo - Version PC}
\end{figure}

\begin{figure}[H]
    \centering
    \includegraphics[width=5cm]{assets/figures/retour_video_natel.jpg}
    \caption{Retour vidéo - Version téléphone portable}
\end{figure}

Le code se trouve dans le fichier \underline{stream.py}, en annexe \ref{stream}.
\newpage
\section{Accès}
L'accès au retour vidéo se fait via une page locale dont l'accès nécessite d'être connecté au WiFi émis par le Rpi4.
La page se trouve à l'adresse statique suivante:
\begin{itemize}
    \item  \url{192.168.0.1:8000/index.html}
\end{itemize}
Accès rapide:

\begin{figure}[H]
    \centering
    \includegraphics[width=3cm]{assets/figures/acces_stream.PNG}
    \caption{QR code - Connexion Stream}
\end{figure}