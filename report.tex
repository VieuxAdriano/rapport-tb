\documentclass[
    iai, % Saisir le nom de l'institut rattaché
    eai, % Saisir le nom de l'orientation
    %confidential, % Décommentez si le travail est confidentiel
]{heig-tb}

\usepackage[nooldvoltagedirection,european,americaninductors]{circuitikz}
\usepackage{pdfpages}
\usepackage{lscape}

\signature{avieux.svg} 

\makenomenclature
\makenoidxglossaries
\makeindex

\addbibresource{bibliography.bib}

\input{nomenclature}
\newacronym{gcd}{GCD}{Plus grand diviseur commun}
\newacronym{lcm}{LCM}{Plus petit multiple commun}
\newacronym{rpi4}{Rpi4}{Rapsberry Pi 4b}

\newglossaryentry{heig-vd}{
    name=HEIG-VD,
    description={Haute École d'Ingénierie et de Gestion du canton de Vaud}
}
\newglossaryentry{hes-so}{
    name=HES-SO,
    description={Haute École Supérieure de Suisse Occidentale}
}
\newglossaryentry{latex}{
    name=latex,
    description={Un langage et un système de composition de documents}
}
\newglossaryentry{maths}{
    name=mathematics,
    description={Les mathematiques sont ce que les mathématiciens fonts}
}

\newglossaryentry{mir}{
    name=MIR,
    description={Infrarouge moyen}
}

\newglossaryentry{ir}{
    name=IR,
    description={Infrarouge}
}

\newglossaryentry{iso}{
    name=ISO,
    description={Organisation internationale de normalisation}
}

\newglossaryentry{wifi}{
    name=WiFi,
    description={wireless fidelity}
}

\newglossaryentry{fps}{
    name=FPS,
    description={frame per seconde}
}

\newglossaryentry{chf}{
    name=CHF,
    description={Confoederatio Helvetica Franc - Franc Suisse}
}

\newglossaryentry{fablab}{
    name=FabLab,
    description={Atelier de fabrication de l'HEIG-VD - salle C08}
}

\newglossaryentry{reds}{
    name=ReDS,
    description={Institut de recherche appliquée et développement de la HEIG-VD}
}

\newglossaryentry{rpi4}{
    name=Rpi4,
    description={Rapsberry Pi 4}
}
% Auteur du document (étudiant-e) en projet de Bachelor
\author{Adriano Vieux}

% Activer l'option pour l'accord du féminin dans le texte
\genre{male}

% Titre de votre travail de Bachelor
\title{Détection de traces d'hydrocarbures sur route par vision industrielle}

% Le sous titre est optionnel
\subtitle{Travail de Bachelor}

% Nom du professeur responsable
\teacher {Prof. P. Bressy (HEIG-VD)}

% Mettre à jour avec la date de rendu du travail
\date{\today}

% Numéro de TB
\thesis{0000}



\surroundwithmdframed{minted}

%% Début du document
\begin{document}
\selectlanguage{french}
\maketitle
\frontmatter
\clearemptydoublepage

%% Requis par les dispositions générales des travaux de Bachelor
\preamble
\authentification

%% Résumé / Résumé publiable / Version abrégée
\begin{abstract}
    % Francais
Le but de ce travail de Bachelor est de mettre en place un système permettant détecter et agir sur les hydrocarbures aux sols par vision industrielle. Actuellement, lorsque les pompiers interviennent pour des cas de fuites d'hydrocarbures sur routes, ils utilisent une remorque-semeuse qu'ils contrôle manuellement afin de déverser du produit absorbant sur les zones à traiter.

La détection est basée sur un éclairage infrarouge et d'un capteur adapté transmettant les captures de la zone à observer vers un boitier de traitement. L'analyse de l'image permet de définir les zones sur lesquelles agir. Un algorithme détermine ensuite où verser le produit absorbant en fonction des données reçues et de la vitesse du véhicule.


\asterism

% English
The purpose of this Bachelor's thesis is to develop a system that can detect and respond to hydrocarbons on the ground using industrial vision. Currently, when firefighters respond to cases of hydrocarbon leaks on roads, they use a manually controlled spreading trailer to spread absorbent material on the affected areas.

The detection is based on infrared illumination and a suitable sensor that transmits captured images of the observed area to a processing unit. The image analysis allows for identifying the areas that need treatment. An algorithm then determines where to apply the absorbent material based on the received data and the vehicle speed.


\end{abstract}

%% Sommaire et tables
\clearemptydoublepage
{
    \tableofcontents
    \let\cleardoublepage\clearpage
    \listoffigures
    \let\cleardoublepage\clearpage
    \listoftables
    \let\cleardoublepage\clearpage
    \listoflistings
}

\printnomenclature
\clearemptydoublepage
\pagenumbering{arabic}

%% Contenu
\mainmatter
\chapter{Introduction}
\newenvironment{listage}{%
    \par%
    \medskip
    \leftskip=4em\rightskip=2em%
    \noindent\ignorespaces}{%
    \par\medskip}

\section{Contexte}
En l'état actuel, lorsqu'un les pompiers doivent intervenir dans le cas d'un incident impliquant des fuites d'hydrocarbures sur route, la répartition
du produit absorbant se fait via une remorque-semoirs dont les sections de dépose du produit sont télécommandées directement par un opérateur.
Cela signifie que durant l'intervention, une personne doit marcher à côté de la remorque afin d'observer et contrôler la répartition du produit, nous pouvons relever les problèmes suivant:
\begin{listage}
    1. L'opérateur doit marcher: Limitation de la vitesse du véhicule, longues distances épuisantes.\newline
    2. La détection se fait à l'oeil nu: L'opérateur doit rester concentrer en permanance au risque de devoir repasser sur certaines zone.\newline
    3. La commande est manuelle: Possibilité d'erreur humaine sur la commande risquant la dépose de surplus de produit ou de devoir repasser sur certaines zone.
\end{listage}
%%if
\section{Citations et bibliographie}
Citer vos sources est essentiel. Avec \texttt{biblatex} vous pouvez facilement citer des articles, des livres ou des sites internet. Toutes les citations dans le texte seront automatiquement regroupées en fin de document dans la section \guillemotleft Bibliographie\guillemotright. Par exemple, citons un article d'Einstein \cite{einstein} ou le livre de Dirac \cite{dirac}.

Parfois il peut être utile d'utiliser un gestionnaire de bibliographie. La communauté académique recommande l'outil \href{https://www.zotero.org/}{Zotero} qui permet de gérer une bibliothèque numérique d'ouvrages et de références numériques. Il permet également de générer une bibliographie compatible avec \LaTeX.

Notez qu'il est très facile d'obtenir l'extrait \texttt{bibtex} depuis des journaux. Sélectionnez \emph{export/citation}. Si vous le pouvez choisissez \texttt{bibtex}. Dans le cas d'un format \texttt{.ris}, utilisez un convertisseur en ligne comme \href{http://www.bruot.org/ris2bib/}{ris2bib}.

\section{Adapter votre modèle}
Ce document n'est qu'un modèle ayant pour but de revoir les quelques avantages de \LaTeX~ et les fonctionnalités qui pourraient vous être utiles pour rédiger un rapport académique. N'hésitez pas à supprimer les parties inutiles et à adapter ce modèle à vos besoins.
%%fi
%\input{examples.tex}

\chapter{Analyse de l'existant}
\label{chap:exi}
\section{Remorque}
\subsection{Caractéristique}
Mesure, éclairage, encombrement, place disponible, autres spécifications.


\chapter{Recherches}
\section{Propriétés réfléctives des élements à capturer}
Durant ses activités, le système observera principalement les élements suivants:
\begin{enumerate}
    \item Les hydrocarbures
    \item Le béton
    \item Le goudron
\end{enumerate}
On m'informant sur les méthodes de détection déjà existante des hydrocarbures, j'ai constaté que l'intégralité d'entre eux fonctionnent
par ultraviolet (mesure de fluoresence) et par infrarouge. Je vais donc m'informer sur la réaction des éléments susmentionné suivant l'exposition à différentes longueurs d'ondes.
\subsection{Hydrocarbures}
Les principaux hydrocarbures traités par les pompiers durant leurs interventions sont les suivants:
\begin{enumerate}
    \item L'huile de moteur.
    \item L'huile hydraulique (tracteur).
    \item L'essence \cite{TotalEnergies}.
    \item Le diesel \cite{TotalEnergies}.
\end{enumerate}
Les éléments principaux qui les composent sont les suivants:
\begin{itemize}
    \item Les alcènes
    \item Les alcanes
    \item Les hydrocarbures aromatiques
\end{itemize}
Divers travaux \cite{Hydrocarbures} traitent des spectres IR des hydrocarbures, mettant en relation la transmittance des éléments en fonction de la longueur d'onde.
Tirés desdits travaux, les graphiques suivants délivres des informations utiles sur la problèmatique du projet.

\begin{figure}[H]
    \centering
    \includegraphics[height=7cm,angle=90]{assets/figures/alcanes1.png}
    \caption{Alcanes - spectre IR du dodécane \cite{Hydrocarbures}}
\end{figure}

\begin{figure}[H]
    \centering
    \includegraphics[height=6cm,angle=90]{assets/figures/alcanes2.png}
    \caption{Alcanes - spectre IR du triméthyl-pentane \cite{Hydrocarbures}}
\end{figure}

\begin{figure}[H]
    \centering
    \includegraphics[height=5cm,angle=90]{assets/figures/alcanes3.png}
    \caption{Alcanes - spectre IR du cycloalcanes \cite{Hydrocarbures}}
\end{figure}

\begin{figure}[H]
    \centering
    \includegraphics[height=5cm,angle=90]{assets/figures/alcenes1.png}
    \caption{Alcènes - spectre IR du cyclohexène \cite{Hydrocarbures}}
\end{figure}

\begin{figure}[H]
    \centering
    \includegraphics[height=5cm,angle=90]{assets/figures/alcenes2.png}
    \caption{Alcanes - spectre IR du  1-octène à l’état liquide\cite{Hydrocarbures}}
\end{figure}

\begin{figure}[H]
    \centering
    \includegraphics[height=5cm,angle=90]{assets/figures/aromatique.png}
    \caption{Aromatique - spectre IR du o-xylène \cite{Hydrocarbures}}
\end{figure}

\newpage
La transmittance indiquée sur l'axe de l'ordonnée représente l'inverse de l'absorbption, c'est à dire que si la transmittance est faible, la lumière
est "très" absorbée par l'hydrocarbure et inversement, si la transmittance est élevée, la lumière est "peu" absorbée par l'hydrocarbure, la laissant ainsi traverser.
Nous observons un point commun pour les trois types d'hydrocarbures, il s'agit des pics d'absorbptions aux alentours de \underline{3000 \si{\per\centi\metre}}, ce qui correspond à \underline{3300 \si{\nano\metre}}.\\
Pour cette longueur d'onde spécifique, la transmittance globale se trouve entre \underline{5 et 30\%}.

\subsection{Béton}

\subsection{Goudron}
On retrouve des hydrocarbures dans la composition du goudron, ceux-ci étant similaires aux hydrocarbures à détecter, le spectre IR pourrait nous indiquer des problèmes de détection pour la classification ...

\subsection{Bitume}
Des informations sont disponibles certains travaux en ligne \cite{Bitume}. Le graphe suivant met en évidence les pics d'absorbption du bitume.

\begin{figure}[H]
    \centering
    \includegraphics[width=13cm]{assets/figures/bitumeIR.png}
    \caption{Bitume - Spectre IR \cite{Bitume}}
\end{figure}

\subsection{Analyse}
On observe que les diverses élements qui seront vus par la caméra ont une variation du taux d'absorptions face aux longueurs d'onde avoisinant les
3300 \si{\nano\metre}, mais à différents ordres de grandeurs. Là ou les hydrocarbures absorbent jusqu'à 95\% des ondes, les routes en absorbent "seulement" jusqu'à 45\%.
En se basant sur ceci, il devrait être possible de faire une différenciation par l'intermédiaire d'un capteur adapté.


Selon les normes \Gls{iso} 20473:2007 \cite{ISO}, cette longueur d'onde se trouve parmi les \Gls{mir}.
\begin{figure}[H]
    \centering
    \includegraphics[width=13cm]{assets/figures/gamme_infra.png}
    \caption{Classification des \Gls{ir} selon les normes \Gls{iso} 20473:2007 \cite{ISO}}
\end{figure}

\section{Test laboratoire}
\subsection{Objet du test}
Vérifier que ça fonctionne dans l'IR
\subsection{Matériel utilisé}
Faire tableau
\subsection{Résultats}
Mettre des jolies images

\chapter{Sélection du matériel et des technologies}
\section{Capture et traitement d'image}
\subsection{Eclairage}
La sélection de l'éclairage va dépendre des points suivants:
\begin{listage}
    1. La longueur d'onde nécessaire.\\
    2. L'intensité lumineuse nécessaire.\\
    3. La taille du champ à "éclairer".
\end{listage}
Le premier point s'est défini durant la phase de recherche, l'idéal serait d'avoir une source émettant une onde de 3300 \si{\nano\metre}.

\subsection{Caméra}
Détail..
\subsection{Système embarqué de traitement}
Détail..

\section{Système de commande}
\subsection{Actionneurs de la télécommande}
Détails
\subsection{Pilotage des actionneurs}
Détails
\section{Retour vidéo}
\subsection{Page web?}
Détails
\section{Communication}


\chapter{Système de vision}
\section{Tests préliminaires}
Le but des tests préliminaires est de vérifier que les analyses effectuées dans la phase de recherche sont applicables et nous retournent
des résultats exploitables. Ils permettent également de se rendre compte si le système imaginé lors de la sélection du matériel est en accord avec
ce qui est réalisable.
\subsection{Matériel}
Au moment des tests, les élements suivants étaient à ma disposition :
\begin{itemize}
    \item 1x Rapsberry Pi 4b - 4Go de RAM.
    \item 1x Pi caméra module 3 NoIR Wide.
    \item 20x leds IR 830nm.
    \item 20x leds IR 850nm.
    \item 20x leds IR 880nm.
    \item 1x morceau de route.
    \item 1l. Huile de moteur neuf - 15W-40.
\end{itemize}

\subsection{Eclairage}
L'éclairage utilisé dans le cadre des tests préliminaires consiste en deux rails de leds IR montées en ligne sur deux veroboards.
Le montage a été effectué selon le schéma suivant:
\begin{figure}[H]
    \centering
    \includegraphics[width=13cm]{assets/figures/schema_leds1.png}
    \caption{Eclairage de test - Schéma électrique de la rangée de led - Schéma modifié de: https://www.sonelec-musique.com/electronique_realisations_alim_led.html}
\end{figure}
Les veroboards sont pratiques à manipuler, il est possible se les tenir avec des étaux afin de faire varier les positions durant les tests. Une fois monté,
le résultat est le suivant:
\begin{figure}[H]
    \centering
    \includegraphics[width=13cm]{assets/figures/rail_led1.jpg}
    \caption{Eclairage de test - Rail de leds 1}
\end{figure}

\begin{figure}[H]
    \centering
    \includegraphics[width=8cm]{assets/figures/rail_led2.jpg}
    \caption{Eclairage de test - Rail de leds 2}
\end{figure}
\subsection{Capture}
La capture d'acquisition des images de tests préliminaires se fait via la caméra sélectionnée durant la phase de décision, un Rpi4 ainsi
qu'un petit script configurant la caméra et enregistrant l'image.
\subsection{Images}
Vu depuis la caméra, nous avons la scène suivante:

\begin{figure}[H]
    \centering
    \includegraphics[width=13cm]{assets/figures/camera_vue_couleur1.png}
    \caption{Capture de test - Vue de la scène en couleur}
\end{figure}

Au moment d'effectuer les tests de sensibilités aux IR, tout le matériel n'était pas encore arrivé, notement le filtre de l'objectif ne
laissant passer que les infrarouges. J'ai donc effectué les captures qui vont suivre dans les conditions suivantes:
\begin{itemize}
    \item Lumières éteintes dans la pièce.
    \item Deux lignes de 10 leds IR éclairant en direction de la tâche d'huile de moteur.
    \item Protection contre les lumières parasites du couloir (carton).
    \item Auto-focus sur le centre de l'image
    \item Temps d'exposition: 5000[ns]. (déterminé expérimentalement avec plusieurs captures)
\end{itemize}
Avec ce setup, j'ai effectué plusieurs captures en faisant varier la position de l'éclairage par rapport à la tâche d'huile et la caméra.
J'ai obtenu différents résultats, plus ou moins utilisables.


Ci-dessous, un exemple d'éclairage orienté "contre" la caméra selon la figure suivante:
\begin{figure}[H]
    \centering
    \includegraphics[width=13cm]{assets/figures/eclairage_contre_camera.png}
    \caption{Schéma de capture - Eclairage contre la camera}
\end{figure}


\begin{figure}[H]
    \centering
    \includegraphics[width=13cm]{assets/figures/eclairage_face1.png}
    \caption{Capture de test - éclairage de face 1}
\end{figure}
\begin{figure}[H]
    \centering
    \includegraphics[width=13cm]{assets/figures/eclairage_face2.png}
    \caption{Capture de test - éclairage de face 2}
\end{figure}

On observe que la route et l'huile reflètent les leds IR, à l'oeil nu la différence est notable, mais l'analyse avec un soft peut s'avérer compliquée.\\
Ci-dessous, un exemple d'éclairage orienté perpendiculairement à la route selon la figure suivante:
\begin{figure}[H]
    \centering
    \includegraphics[width=13cm]{assets/figures/eclairage_perpendiculaire.png}
    \caption{Schéma de capture - Eclairage perpendiculaire à la route}
\end{figure}

\begin{figure}[H]
    \centering
    \includegraphics[width=13cm]{assets/figures/eclairage_perpendiculaire1.png}
    \caption{Capture de test - Eclairage perpendiculaire 1}
\end{figure}

\begin{figure}[H]
    \centering
    \includegraphics[width=13cm]{assets/figures/eclairage_perpendiculaire2.png}
    \caption{Capture de test - Eclairage perpendiculaire 2}
\end{figure}

\begin{figure}[H]
    \centering
    \includegraphics[width=13cm]{assets/figures/eclairage_perpendiculaire3.png}
    \caption{Capture de test - Eclairage perpendiculaire 3}
\end{figure}
On observe que la route diffuse, dans une moindre mesure, l'éclairage des leds IR, là ou l'huile semble absorber les rayonnements. On arrive
assez facilement à différencier le clair de la route et le foncé de l'huile. La mise en place d'un software de détection est envisageable avec
un éclairage basé sur ce schéma. A noté que les tests ont été effectué avec plusieurs gammes de leds (830nm, 850nm et 880nm), la différence
sur le retour image n'est pas très grandes, mais les leds 850nm permettent de mieux différencier la route de l'huile.

[Faire un test avec l'éclairage derrière la caméra]

\section{Installation}
\subsection{Eclairage}

\subsection{Caméra}

\section{Programme}
\subsection{Traitement de l'image}
\subsection{Communication}


\chapter{Contrôle de la télécommande}
\section{Tests préliminaires}
Le but des tests préliminaires est de vérifier que les analyses effectuées dans la phase de recherche sont applicables et si la mise en
place du système imaginé lors de la phase de sélection du matériel est aussi efficace qu'espéré.

Il a été décidé de contrôler la télécommande via un système de préhenseur d'aimant de levage. Les données disponibles de la télécommande
ne stipule pas la force nécessaire à l'activation d'un de ses boutons, plusieurs préhenseurs sélectionné dans le stock du \Gls{fablab}
sont utilisés pour définir la force nécessaire à l'activation des touches.

\begin{itemize}
    \item Préhenseur 6VDC 4W, 0.4N -> 2N.
    \item Préhenseur 12VDC 7W, 0.6N -> 11N.
\end{itemize}

Ils ont été installé successivement sur une pièce en alu usinée spécialement pour l'essaie, permettant aux préhenseurs de se déployer
sans que la télécommande puisse se déplacer.

[Mettre une belle photo]

\begin{table}[h]
    \begin{center}
        \caption{Résultats du test de préhension}
        \begin{tabular}{|c|l|}
            Elements       & Boutons \\ \hline
            Préhenseur 2N  & OUI-NON \\
            Préhenseur 11N & OUI-NON \\
        \end{tabular}
    \end{center}
\end{table}\\

\section{Supports}

\subsection{Support pour jeu de préhenseur magnétique}
\subsection{Support de la télécommande}

\section{Programme}
\subsection{Pilotage des vérins}
\subsection{Communication}

\chapter{Installation sur le semoir}
\section{Liste des éléments à installer}
\begin{itemize}
    \item Deux rails de leds IR.
    \item Un boitier étanche contenant: Le Rpi4, la caméra, le transformateur 12VDC-5VDC, le régulateur 1.5V et le fusible d'entrée.
    \item Tiroir pour la télécommande avec les préhenseurs.
    \item Alimentation 13 broches.
\end{itemize}
\section{Disposition prévue}
Comme indiqué dans le chapitre \ref{chap:exi}, nous avons possibilité d'utiliser le support perforé ainsi que la zone libre sur le timon.
Le principe est le suivant, fixer le boitier étanche à une plaque d'aluminium qui elle même sera fixer sur le support, ainsi, il sera plus simple
de placer le boitier selon les besoins. Le boitier sera alimenté à l'aide d'une rallonge 13 broches modifiées, permettant d'utiliser le 12[V] de la voiture.
\begin{figure}[H]
    \centering
    \includegraphics[height=6cm, angle=90]{assets/figures/alim.jpg}
    \caption{Montage prévu - Alimentation}
\end{figure}
La camera et son filtre seront placés dans le boitier et plaqués contre la façade transparente, de manière à observer la route entre le semoir
et le pick-up.

Les rails de leds IR seront placés de part et d'autre du timon à 80 \si{\centi\metre} du semoir en direction du pick-up pour "éclairer" le champ
de vue de la caméra. \textbf{Dans l'idéal, la fixation permettra aux deux rails de se replier sur le timon lorsque l'éclairage ne sera pas utilisé,
    cependant il n'est pas certain que ce sera effectué durant la période dédiée au TB.}

Le tiroir permettant de contrôler la télécommande serait placé sur le boitier étanche, celui-ci s'ouvrant de face, ça ne posera pas problème.

L'aimant et le capteur Reed permettant de mesurer la vitesse seront placés sur l'essieu, à proximité du boitier central.

Les illustrations suivantes permettent de se faire une idée plus précise de ce qui est prévu.

\begin{figure}[H]
    \centering
    \includegraphics[height=6cm]{assets/figures/montage1.png}
    \caption{Montage prévu - Vue de côté}
\end{figure}

\begin{figure}[H]
    \centering
    \includegraphics[height=5cm]{assets/figures/montage2.png}
    \caption{Montage prévu - Vue de face}
\end{figure}
\newpage
\begin{figure}[H]
    \centering
    \includegraphics[height=7cm]{assets/figures/montage3.png}
    \caption{Montage prévu - Vue du dessus}
\end{figure}

\begin{figure}[H]
    \centering
    \includegraphics[height=4cm]{assets/figures/montage4.png}
    \caption{Montage prévu - Mesure de vitesse}
\end{figure}
\section{Installation réel}
\subsection{Eclairage}
Le timon ne pouvant pas être percé, j'ai imaginé la fixation suivante pour le maintient des barres de leds.

\begin{figure}[H]
    \centering
    \includegraphics[width=13cm]{assets/figures/fixation_eclairage_face.PNG}
    \caption{Fixation leds - Vue de face}
\end{figure}

\begin{figure}[H]
    \centering
    \includegraphics[width=13cm]{assets/figures/fixation_eclairage_cote.PNG}
    \caption{Fixation leds - Vue de côté}
\end{figure}

Le principe est de visser les deux rails à une première plaque en aluminium, puis visser celle-ci à une seconde plaque placée au dessus du timon,
de manière à le serrer. Cette solution ne permet pas de replier les leds, mais c'est suffisamment solide pour effectuer des tests de détection.

Ci-dessous, le montage réalisé:

\begin{figure}[H]
    \centering
    \includegraphics[height=13cm]{assets/figures/fixation_eclairage1.PNG}
    \caption{Fixation leds 1}
\end{figure}

\begin{figure}[H]
    \centering
    \includegraphics[width=9cm]{assets/figures/fixation_eclairage2.PNG}
    \caption{Fixation leds 2}
\end{figure}

Après essai entre 5 et 10 km/h sur le parking de l'école (surface pas forcément très régulière), la fixation reste en place et grâce à la structure du rails de leds, le tout vibre très peu.
\subsection{Boitier}
Ci-dessous la disposition des éléments du boitier:

\begin{figure}[H]
    \centering
    \includegraphics[width=9cm]{assets/figures/boitier_montage.jpg}
    \caption{Fixation dans le boitier}
\end{figure}

De gauche à droite, on retrouve le Rpi4, le régulateur 1.5 V, l'alimentation 5 V, le fusible d'entrée et,
en bas à droite, la caméra équipée de son filtre. Ces éléments sont montés sur le rail DIN dont les supports ont été imprimés
en 3D, inspirés du modèle suivant: \cite{support_Rpi_3D}.

Ci-dessous, le boitier est fixé à sa plaque d'adaptation, elle-même fixée au support du semoir.

\begin{figure}[H]
    \centering
    \includegraphics[height=9cm]{assets/figures/boitier_adaptation.jpg}
    \caption{Fixation sur le semoir}
\end{figure}

\subsection{Vue globale}
Ci-dessous un aperçu de l'installation comprenant le boitier et son support, l'éclairage et sa fixation ainsi que les câbles d'alimentations.
\begin{figure}[H]
    \centering
    \includegraphics[height=9cm]{assets/figures/installation_globale.jpg}
    \caption{Installation entière}
\end{figure}

\chapter{Retour vidéo}
\section{Communication}
Il a été décidé durant la phase de recherche que l'affichage se fera via une page locale qui recevra les images via un réseau WiFi local
hébergé par le Raspberry Pi \cite{soren_how-_2013}. Les étapes pour y arriver sont les suivantes:
\begin{enumerate}
    \item Flasher avec l'OS Raspberry (en utilisant Raspberry imager ou BalenaCloud).
    \item Être connecté à internet.
    \item Télécharger le package 'dnsmasq': \textbf{sudo apt-get install dnsmasq}.
    \item Atteindre le fichier suivant: \textbf{sudo nano /etc/wpa\char`_supplicant/wpa\char`_supplicant.conf}.
    \item Y ajouter le code du listing \ref{wpa}, sauver + quitter: ctrl+O, puis ctrl+X.
    \item Atteindre le fichier suivant: \textbf{sudo nano /etc/network/interfaces}.
    \item Y remplacer le code du listing \ref{interfaces}, sauver + quitter: ctrl+O, puis ctrl+X.
    \item Atteindre le fichier suivant: \textbf{sudo nano /etc/dnsmasq.conf}.
    \item Y remplacer le code du listing \ref{dnsmasq}, sauver + quitter: ctrl+O, puis ctrl+X.
    \item Redémarrer le Rpi4 avec: \textbf{sudo reboot}.
    \item Couper la connexion réseau avec: \textbf{sudo ifdown wlan0}.
    \item Accéder à la liste des réseaux avec: \textbf{sudo wpa\char`_cli -i wlan0 list\char`_networks}, puis saisir le réseau avec: \textbf{sudo wpa\char`_cli -i wlan0 select\char`_network 1} (exemple figure \ref{networks_list})
    \item Sauver la nouvelle configuration avec \textbf{sudo wpa\char`_cli -i wlan0 save\char`_config}.
    \item Relancer la connexion réseau avec: \textbf{sudo ifup \underline{wlan0=ap0}}.
\end{enumerate}
\begin{listing}[ht]
    \inputminted{makefile}{assets/figures/wpa_supplicant.make}
    \caption{Configuration wpa\char`_supplicant \label{wpa}}
\end{listing}

\begin{listing}[ht]
    \inputminted{makefile}{interfaces.make}
    \caption{Configuration de l'interface réseau \label{interfaces}}
\end{listing}

\begin{listing}[ht]
    \inputminted{makefile}{dnsmasq.make}
    \caption{Configuration dnsmasq \label{dnsmasq}}
\end{listing}

\begin{figure}[H]
    \centering
    \includegraphics[width=13cm]{assets/figures/network_list.PNG}
    \caption{Liste de networks \label{networks_list}}
\end{figure}

Avec l'ajout de ces lignes, le Rpi4 émet un réseau local, pour que l'émission se fasse dès le démarrage de la carte,
il faut ouvrir le fichier suivant avec : \textbf{sudo nano /home/tb/.bashr} et y ajouter les lignes de codes des étapes \textbf{11 et 14}.

Il est possible de s'y connecter avec n'importe quel
type d'appareil. Le nom du réseau et le mot de passe sont paramétrables, actuellement, c'est:
\begin{itemize}
    \item Nom du réseau: \textbf{SemoirSDIS}
    \item Mot de passe: \textbf{123456789}
\end{itemize}\\
À noter qu'il est obligatoire de s'y connecter pour avoir le retour vidéo.

\begin{figure}[H]
    \centering
    \includegraphics[width=3cm]{assets/figures/acces_wifi.PNG}
    \caption{QR code - Connexion WiFi local}
\end{figure}

\newpage
\section{Affichage}
Il existe des codes sources disponibles en ligne permettant d'afficher le retour caméra basé sur le package Picamera2 \cite{picamera2}.
Je me suis inspiré de celui-ci pour mon affichage: \cite{code_camera}.

L'utilisation de \textbf{Picamera2} nécessite l'installation de dépendances:
\begin{itemize}
    \item \textbf{sudo apt-get install libatlas-base-dev}.
    \item \textbf{sudo apt install -y python3-picamera2}.
\end{itemize}

\begin{figure}[H]
    \centering
    \includegraphics[width=13cm]{assets/figures/retour_video_pc.png}
    \caption{Retour vidéo - Version PC}
\end{figure}

\begin{figure}[H]
    \centering
    \includegraphics[width=5cm]{assets/figures/retour_video_natel.jpg}
    \caption{Retour vidéo - Version téléphone portable}
\end{figure}

Le code se trouve dans le fichier \underline{stream.py}, en annexe \ref{stream}.
\newpage
\section{Accès}
L'accès au retour vidéo se fait via une page locale dont l'accès nécessite d'être connecté au WiFi émis par le Rpi4.
La page se trouve à l'adresse statique suivante:
\begin{itemize}
    \item  \url{192.168.0.1:8000/index.html}
\end{itemize}
Accès rapide:

\begin{figure}[H]
    \centering
    \includegraphics[width=3cm]{assets/figures/acces_stream.PNG}
    \caption{QR code - Connexion Stream}
\end{figure}

\chapter{Conclusion}
\section{Ressenti personnel}

\section{Conclusion technique}

\section{Remercicement}
Je tenais à remercier les différentes personnes qui m'ont entourés, accompagnés et aidés durant ce projet.

Je remercie M. Tristan Lieberherr, qui m'a partagé ses connaissances sur les Rapsberry Pi et sur la configuration des réseaux WiFi locaux.

Je remercie M. Nicolas Tzaut, qui m'a très gentiment céder un morceau de route provenant de sa démolition, grâce auquel j'ai pu effectué
des tests de capture d'image en laboratoire très rapidement après le début du projet.

Et je remercie finalement M. Pierre Bressy, qui m'a suivi tout du long du projet. Il a su me... et me donner de nouvelles idées lorsque
j'étais bloqué.
\vfil
\hspace{8cm}\makeatletter\@author\makeatother\par
\hspace{8cm}\begin{minipage}{5cm}
    %%if
    % Place pour signature numérique
    \printsignature
    %%fi
\end{minipage}

%remercier Tzaut & Fils Sa pour les matériaux, Tristan pour les apports Rpi4

\clearpage
\printbibliography

\appendix
\appendixpage
\addappheadtotoc

%%if
\chapter{Première annexe}

Les annexes n'ont pas un contenu \underline{normatif} mais \underline{descriptif}. Tout contenu annexé ne doit pas être nécessaire à la bonne compréhension du travail.

Les annexes contiennent généralement :

\begin{itemize}
    \item les dessins mécaniques (mises en plan);
    \item les schémas électriques détaillés;
    \item des photographies du projet;
    \item des scripts et des extraits de code source;
    \item des documents techniques \pex \emph{datasheet};
    \item des développements mathématiques.
\end{itemize}
\section{Sous section}
\lipsum[1]

\chapter{Datasheets}
\section{Led infrarouge 850nm}
\includepdf[pages=-]{assets/pdf/datasheet_led.pdf}
\section{PiCamera module 3 NOiR Wide}
\includepdf[pages=-]{assets/pdf/datasheet_camera.pdf}

\chapter{Programmes}

\section{Test de captures \label{test_process}}
\includepdf[pages=-]{assets/code/test_process.py.pdf}
\section{Capture images de références \label{img_ref}}
\includepdf[pages=-]{assets/code/cap_img_ref.py.pdf}

\section{Libraire de fonctions \label{libraire_det}}
\includepdf[pages=-]{assets/code/detection_lib.py.pdf}

\section{Main \label{main}}
\includepdf[pages=-]{assets/code/main.py.pdf}

\section{Stream \label{stream}}
\includepdf[pages=-]{assets/code/stream.py.pdf}
%%fi

\let\cleardoublepage\clearpage
\backmatter

\label{glossaire}
\printnoidxglossary
\label{index}
\printindex

% Le colophon est le dernier élément d'un document qui contient des notes de l'auteur concernant la mise en page et l'édition du document : il est parfaitement optionnel.
%\input{colophon.tex}
\end{document}